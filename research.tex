\cvsection{Research}

\vspace{1mm}

\cvsubsection{Software Systems}

\begin{cvbib}

\cvbibitem{S16}{sw-mist}

\end{cvbib}

%
%<publication>
%  <pubkey>[<a name="S15">S15</a>]</pubkey>
%  <pubdata>
%    <i>A Simple Ushahidi Java API</i>.
%    <!--{<https://github.com/CSG-CS2/simple-ushahidi-api>}-->.
%    2013-present.
%  </pubdata>
%</publication>
%
%<publication>
%  <pubkey>[<a name="S14">S14</a>]</pubkey>
%  <pubdata>
%    <i>gigls</i> - The Glimmer Improved Gimp Library for Scripting.  Extensions 
%    to the GNU Image Manipulation Program (GIMP) to support novice
%    programmers.
%    <!--{<https://github.com/GlimmerLabs/gigls>}-->.
%    2012-present.
%  </pubdata>
%</publication>
%
%<publication>
%  <pubkey>[<a name="S13">S13</a>]</pubkey>
%  <pubdata>
%    <i>louDBus</i>.  A D-Bus client library for the Racket programming
%    language.
%    <!--{<https://github.com/GlimmerLabs/louDBus>}-->.
%    2012-2014.
%  </pubdata>
%</publication>
%
%<publication>
%  <pubkey>[<a name="S12">S12</a>]</pubkey>
%  <pubdata>
%    <i>IAScript/MediaScript</i>.  An interactive scripting system with
%    support for multiple languages (including Scheme and Python) and 
%    extensions for multiple platforms (including GIMP and InkScape).  
%    2007-present.  Described in
%    [<a href="#B25">B25</a>],
%    [<a href="#B26">B26</a>],
%    [<a href="#D7">D7</a>],
%    [<a href="#D8">D8</a>],
%    [<a href="#D9">D9</a>], and
%    [<a href="#E20">E20</a>].
%    Successor to [<a href="#S10">S10</a>].
%  </pubdata> 
%</publication>
%
%<publication>
%  <pubkey>[<a name="S11">S11</a>]</pubkey>
%  <pubdata><i>Phoenix</i>.  A nonlinear video editing system which emphasizes
%  script-based editing using Scheme.  2006-2009.  A preliminary version
%  is described in
%  [<a href="#D5">D5</a>].</pubdata>
%</publication>
%
%<publication>
%  <pubkey>[<a name="S10">S10</a>]</pubkey>
%  <pubdata><i>DrFu + Higher Order Graphics</i>.  A collection of extensions to
%  the GNU Image Manipulation Program (GIMP) to support higher-order
%  manipulation of images.  2005-2007.  
%  Precursor to [<a href="#S12">S12</A>].</pubdata>
%</publication>
%
%<publication>
%  <pubkey>[<a name="S9">S9</a>]</pubkey>
%  <pubdata><i>Project Clio</i>.  A system for tracking student
%  use of course webs.  1997-2006.  (A prototype of the system is
%  described in [<a href="#B13">B13</a>]; visualization tools are
%  described in [<a href="#B19">B19</a>], [<a href="#B22">B22</a>],
%  and [<a href="#D2">D2</a>]; a data-mining tool is described in [<a
%  href="#D1">D1</a>].)</pubdata>
%</publication>
%
%<publication>
%  <pubkey>[<a name="S8">S8</a>]</pubkey>
%  <pubdata><i>Web Raveler</i>.  A system for customizing views of
%  the Web.  Includes an annotation system, a trail system, and a
%  number of other page mediators.  1997-2006.  (The original system
%  is described in in [<a href="#B13">B13</a>]; a revised system is
%  described in [<a href="#B18">B18</a>], [<a href="#D3">D3</a>], and
%  [<a href="#B23">B23</a>]; a prototype annotation system is described in
%  [<a href="#B14">B14</a>]; a prototype trail system is described in [<a
%  href="#B17">B17</a>]; a prototype link summary system is described in
%  [<a href="#B20">B20</a>].  </pubdata>
%</publication>
%
%<publication>
%  <pubkey>[<a name="S7">S7</a>]</pubkey>
%  <pubdata><i>Site Weaver</i>.  A collection of tools for building
%  hypermedia documents, particularly course webs.  Includes a
%  shorthand system for converting documents between formats.
%  1997-2006.  (The original shorthand system is described in [<a
%  href="#B11">B11</a>].)</pubdata>
%</publication>
%
%<publication>
%  <pubkey>[<a name="S6">S6</a>]</pubkey>
%  <pubdata><i>CourseWeaver</i>.  A hypermedia system for designing
%  and building reconfigurable course webs. Automates many steps in
%  the construction and reconstruction of course webs.  Also permits
%  multiple <Q>views</Q> of the same data (web-based, stand-alone, or
%  grouped for printing) and allows the instructor to organize the same
%  data in many ways.  1995-1996.  (Described in [<a href="#B9">B9</a>];
%  precursor to [<a href="#S7">S7</A>].)</pubdata>
%</publication>
%
%<publication>
%  <pubkey>[<a name="S5">S5</a>]</pubkey>
%  <pubdata><i>Electronic Conference Submissions Server</i>.  Developed in
%  Perl for STOC'95 and other conferences.  1994-1998.  Used for a number
%  of conferences, including ACM Symposium on the Theory of Computation,
%  IEEE Symposium on Foundations of Computer Science, and SIAM Symposium
%  on Discrete Algorithms In 1998, it was used for at least a dozen
%  conferences.</pubdata>
%</publication>
%
%<publication>
%  <pubkey>[<a name="S4">S4</a>]</pubkey>
%  <pubdata><i>DAGS Multimedia Proceedings</i>.  Complete redesign and reimplementation of sophisticated interface for electronic proceedings that include both talks (audio, video, and slides) and papers.  1994-1996.  (Described in [<a href="#A4">A4</a>], [<a href="#B5">B5</a>] and [<a href="#F6">F6</a>]).</pubdata>
%</publication>
%
%<publication>
%  <pubkey>[<a name="S3">S3</a>]</pubkey>
%  <pubdata><i>Software for Introductory Computer Science</i>. Includes assembly code "interpreter," HTML editor, animated sorting and searching algorithms, HyperCard stack templates, HyperCard-based hypertext guide to computer jargon, sample games, and an electronic blackboard.  Written in HyperCard and JavaScript.  1993-1998.</pubdata>
%</publication>
%
%<publication>
%  <pubkey>[<a name="S2">S2</a>]</pubkey>
%  <pubdata><i>Tours: A System for Term-Based I/O</i>.  A generic incremental, demand-driven, term-based communication system.  Used as new I/O system for Equational programming system and provides interoperability between declarative and imperative programs.  1990-1993.  Described in [<a href="#F4">F4</a>].</pubdata>
%</publication>
%
%<publication>
%  <pubkey>[<a name="S1">S1</a>]</pubkey>
%  <pubdata><i>Equational Compiler for Sun Workstations</i>.  Reimplementation of compiler for equation-based declarative language.  Written in T (a LISP dialect) to generate 68000 code.  1987-1988.  </pubdata>
%</publication>
%
%</publications>

\cvsubsection{External Funding}

\begin{cventries}

\cventryB{Extending Introductory Computer Science with Algorithmic Multimedia}{}{July 1998 to June 2000.  Extended to June 2001.}
\cventryC{NSF Instrumentation and Laboratory Improvement Grant DUE \#98-50546}
\cventryC{Samuel A. Rebelsky (PI), John D. Stone (Co-PI), and Henry M. Walker (Co-PI)}
\cventryC{\$33,600}

\cventryB{Trailblazing Tools for the World Wide Web}{}{September 1998 to May 1999}
\cventryC{The CRA Collaborative Research Environment for Women in Undergraduate Computer Science and Engineering Program}
\cventryC{Samuel A. Rebelsky (Faculty Sponsor).  Rachel Heck, Sarah
  Luebke, Weichao Ma, and Hilary Mason (Students)}
\cventryC{\$3,400}

\cventryB{Untitled Grant}{}{Winter 1999}
\cventryC{Smarter Kids Foundation}
\cventryC{Samuel A. Rebelsky (PI)}
\cventryC{Approximately \$2000}

\cventryB{Travel Grant}{}{Summer 2000}
\cventryC{Iowa Computer Science Preparing Future Faculty Consortium using funds from a National Science Foundation program.}
\cventryC{Supported student travel to EdMedia 2000 World Conference
  on Educational Multimedia and Hypermedia.}
\cventryC{Approximately \$5000.}

\cventryB{Faculty Career Enhancement (FaCE)}{}{May 2006--August 2006}
\cventryC{Associated Colleges of the Midwest (ACM)}
\cventryC{Provided \$3000 to support attendance at SIGGRAPH and related conferences.}

\cventryB{Reformulating Media Computation with Functional Programming, Scripting, and Design Principles}{}{May 2007 to July 2009}
\cventryC{National Science Foundation Course, Curriculum, and Laboratory Improvement (CCLI) grant 0633090}
\cventryC{Samuel A. Rebelsky (PI), Janet Davis (Co-PI), Matthew Kluber (Co-PI)}
\cventryC{\$148,763}

\end{cventries}
